
\subsection{Changes to TOSSIM}

The changes to modify TOSSIM code to interpret the loss in the \emph{.nss} file as
packet level loss were made in the TinyOs code tree in the tinyos-1.1.10/tos/platform/pc/packet/TosssimPacketM.nc
file.\newline

\noindent
All the experiments for this document were carried out on sb06.eecs.harvard.edu in the
\emph{local geetika Fits-20584} directory.

\subsection{Sequence for Conducting Experiments}

\begin{enumerate}\addtolength{\itemsep}{-0.5\baselineskip}
\item Generate experiment file(s). The format of an experiment file is as follows: 
Datadir FFConstant TopologyFile NumMotes Time TopologyTag RandomSeed Makefile IgnorePeriod ProcessDelay SendDelay 
Use \emph{createExpFile.pl} in the tools directory.
If repeating the experiment for multiple runs, change the random seed.
\item Run the experiment:
\emph{..tools doExperiment.pl  yourExpFile expNum.txt ..data local geetika tinyos 1.1.10 tos-NUM}
The shell script doexpt will do a number of these by changing the expNum variable.
Don't worry about recompiling the system. Doexperiment.pl will do that for you as long as you tell
it what diretory to compile in. Make sure to give the correct directory name: \emph{Eg local geetika tinyos-1.1.10 tosX}
\item Run the metric computation scripts on the experiments
\emph{.doIt.pl ..data dirName-expNum .. data dirNameexpNum dirNameexpNumresFile.txt}
The shell script doprocessing will do these for all runs by changing the expNum variable.
\item Now average the data over multiple runs and compute error bars by running avgMetricMultRuns.pl
from the tools directory. This can compute the percent-synched cases, time to sync, and 50th and 90th percentile
group spreads by passing different options to this script.
\end{enumerate}



