\section{Introduction}

Computer scientists have often looked to nature for inspiration.
Researchers studying distributed systems have long envied, and
attempted to duplicate, the fault-tolerance and decentralized control
achieved in natural systems.  Those of us studying sensor networks
also have every reason to be envious.  Designing software coordinating
the output of a collection of limited devices frequently feels as
frustrating as orchestrating the activity of a colony of stubborn
ants, or guiding a school of uncooperative fish.  And yet ant colonies
complete difficult tasks, schools of fish navigate the sea, and swarms
of fireflies stretching for miles can pulse in perfect unison, all
without centralized control or perfect individuals. The spontaneous
emergence of synchronicity --- for example, fireflies flashing in
unison or cardiac cells firing in synchrony --- has long attracted the
attention of biologists, mathematicians and computer scientists.

%% SIMULTANEOUS COLLECTIVE ACTION

Synchronicity is a powerful primitive for sensor networks. We define
synchronicity as the ability to organize {\em simultaneous collective
action} across a sensor network. Synchronicity is not the same as time
synchronization: the latter implies that nodes share a common notion
of time that can be mapped back onto a real-world clock, while the
former only requires that nodes agree on a firing period and phase.
The two primitives are complementary: nodes with access to a common
time base can schedule collective action in the future, and
conversely, nodes that can arrange collective action can establish a
meaningful network-wide time base. However, the two primitives are also
independently useful. For example, nodes within a sensor network may
want to compare the times at which they detected some event. This task
requires a notion of global time, however it does not require
real-time coordination of actions. 

Similarly, synchronicity by itself can be extremely useful as a sensor
network coordination primitive. A commonly-used mechanism for limiting
energy use is to carefully schedule node duty cycles so that all nodes
in a network (or a portion of the network) will wake up at the same
time, sample their sensors, and relay data along a routing path to the
base station. Coordinated communication scheduling has been used both
at the MAC level~\cite{s-mac} and in multi-hop routing
protocols~\cite{stem} to save energy. Synchronicity can also be
used to coordinate sampling across multiple nodes in a network,
which is especially important in applications with high data
rates. Previous work on seismic analysis of
structures~\cite{glaser-smart-buildings}, shooter
localization~\cite{shooter-localization}, and volcanic
monitoring~\cite{volcano-ewsn05} could use such a primitive and avoid
the overhead of maintaining consensus on global time until
absolutely necessary.

In this paper, we present a biologically-inspired distributed
synchronicity algorithm implemented on TinyOS motes. This algorithm is
based on a mathematical model originally proposed by Mirollo and
Strogatz to explain how neurons and fireflies spontaneously
synchronize~\cite{strogatz}. This seminal work proved that a very
simple reactive node behavior would always converge to produce global
synchronicity, irrespective of the number of nodes and starting
times. Recently Lucarelli and Wang~\cite{lucarelli04} demonstrated
that this result also holds for multi-hop topologies, an important
contribution towards making the model feasible for sensor networks.

The firefly-inspired synchronization described by Mirollo and Strogatz
has several salient features that make it attractive for sensor
networks. Nodes execute very simple computations and interactions, and
maintain no internal state regarding neighbors or network topology. As
a result, the algorithm robustly adapts to changes such as the loss
and addition of nodes and links \cite{lucarelli04}. The synchronicity
provably emerges in a completely decentralized manner, without any
explicit leaders and irrespective of the starting state.

However, implementing this approach on wireless sensor networks still
presents significant obstacles. In particular, the previous
theoretical work assumes {\em instantaneous} communication between
nodes.  In real sensor networks, radio contention and processing
latency lead to significant and unpredictable communication
latencies. Earlier work also assumes non-lossy radio links, identical
oscillator frequencies, and arbitrary-precision floating-point
arithmetic which are unrealistic in current sensor networks.

We present the {\em reachback firefly algorithm} (RFA) that accounts for
communication latencies, by modifying the original firefly model to
allow nodes to use information from the past to adjust the future
firing phase. We evaluate our algorithm in three ways: theory,
simulation and implementation. We present theoretical results to prove
the convergence of our algorithm in simple cases and predict the
impact of parameter choice. Next we leverage TOSSIM, the TinyOS
simulator, to explore the behavior of the algorithm over a range of
parameter values, varying numbers of nodes, and different
communication topologies. These simulation results validate the
theoretical predictions. Finally, we present results from experiments
on a real sensor network testbed. These results demonstrate that our
algorithm is robust in the face of real radio effects and node
limitations.  Our results show that such a decentralized approach can
provide synchronicity to within 100~$\mu$sec on a complex multiple-hop
network with asymmetric and lossy links. To the best of our knowledge,
this work represents the first implementation of firefly-inspired
synchronicity on the MicaZ mote hardware, and demonstrates the ability
of the model to achieve synchronicity given real radio and hardware
limitations.

Our paper is organized as follows. Section \ref{sec-background}
presents related work. In Section \ref{sec-algorithm} we present RFA
in the context of the Mirollo and Strogatz model and describe current
hardware and radio limitations. Sections
\ref{sec-theory}-\ref{sec-motes} present our metrics and theoretical,
simulation and experimental results. We conclude with future work.



