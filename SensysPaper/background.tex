\section{Background and Motivation}
\label{sec-background}

Time synchronization has received a great deal of attention in the
sensor network community. The problem of establishing a consistent global
timebase across a large network, despite message loss and delays, node
failures, and local clock skew, has proven to be very difficult. As
described in the introduction, our goal is not time synchronization,
but rather {\em synchronicity:} the ability for all nodes in the
network to agree on a common period and phase for firing
pulses. Synchronicity can be used to implement time synchronization,
although this requires mapping the local firing cycle to a
global clock, which we leave for future work. 

%% not quite true
%% For example, the Flooding Time Synchronization Protocol (FTSP)~\cite{ftsp}
%% uses multiple rounds of message transmission on each synchronization
%% cycle to accurately estimate clock skew and message delays. In cases
%% where only synchronicity is required, this overhead may not be desirable.

%% mostly moved to introduction
%% Synchronicity without time synchronization is extremely useful as a 
%% sensor network coordination primitive. A commonly-used mechanism for
%% limiting energy use is to carefully schedule node duty cycles so that
%% all nodes in a network, or a portion of the network, will wake up at
%% the same time, sample their sensors, and relay data along a routing
%% path to the base station. Coordinated communication scheduling 
%% has been used both at the MAC level~\cite{t-mac,s-mac} and in 
%% multihop routing protocols~\cite{leach,stem} to save 
%% energy.

%% Synchronicity can also be used to correlate signals across multiple
%% nodes in a network, which is especially important in applications
%% with high data rates. Previous work on seismic analysis of
%% structures~\cite{glaser-smart-buildings,wisan}, shooter
%% localization~\cite{shooter-localization}, and volcanic 
%% monitoring~\cite{volcano-ewsn05} could use a synchronicity primitive, 
%% rather than time synchronization, to establish signal correlation
%% across sensor nodes.

\subsection{Time Synchronization}

%Time synchronization in sensor networks is a well-studied problem.

A number of protocols have been proposed that allow wireless sensor
nodes to agree on a common global timebase. Here we briefly describe
some of the protocols. In Receiver Based Synchronization
(RBS)~\cite{rbs} a reference node broadcasts a message and multiple
receivers within radio range can then agree on a common time base by
exchanging the local clock times at which they received the message.
This protocol avoids the uncertainty of transmission delays by using a
single radio message to simultaneously synchronize multiple receiver
nodes, however it does not apply in multi-hop networks. The
TPSN~\cite{tpsn} protocol works on multi-hop networks by constructing
a spanning tree and then using hop-by-hop synchronization along the
edges to synchronize all nodes to the root. They also introduce MAC
level timestamping to estimate transmission delay. The FTSP
protocol \cite{ftsp}, simplifies the process of multi-hop
synchronization by using periodic floods from an elected root, rather
than maintaining a spanning tree. In the case of root failure, the
system elects a new root node. FTSP also refines the timestamping
process to within microsecond accuracy and provides a method for
estimating clock drift which reduces the need to synchronize
frequently.

Direct comparison of these protocols in terms of synchronization error
is difficult, due to the differences in hardware and evaluation
methodology. FTSP reports a per-hop synchronization error of about
1~$\mu$sec, although the maximum pairwise error is over 65~$\mu$sec in
their testbed.  The mean single-hop synchronization error reported for
TPSN is 16.9~$\mu$sec, compared to 29.1~$\mu$sec for RBS~\cite{tpsn}.
The dynamics of these protocols in terms of robustness to topology
changes and node population have not been widely studied.

\subsection{Biologically-Inspired Synchronicity}

Synchronicity has been observed in large biological swarms where
individuals follow simple coordination strategies. The canonical
example is the synchrony of fireflies observed in certain parts of
southeast Asia~\cite{strogatz}.  The behavior of these systems can be
modeled as a network of {\em pulse-coupled oscillators} where each
node is an oscillator that periodically emits a self-generated pulse.
Upon observing other oscillators' pulses, a node adjusts the phase of
its own oscillator slightly. This simple feedback process results in
the nodes tightly aligning their phases and achieving synchronicity.

Peskin first introduced this model in the context of cardiac pacemaker
cells\cite{peskin75}. Mirollo and Strogatz~\cite{strogatz} provide one
of the earliest complete analytical studies of pulse-coupled
oscillator systems. They proved that a fully-connected (all-to-all)
network of $N$ identical pulse-coupled oscillators would synchronize,
for any $N$ and any initial starting times. Recent work by Lucarelli
and Wang~\cite{lucarelli04} relaxes the all-to-all communication
assumption. Drawing from recent results in multi-agent control, they
derive a stability result based on nearest neighbor coupling and show
convergence in simulation for static and time varying
topologies. Their work demonstrates that the same simple feedback
process works, even when nodes only observe nearest neighbors and
those neighbors may change over time.

Several groups have proposed using pulse-coupled synchronicity to
solve various network problems. Hong and
Scaglione~\cite{tsrbc03,hcs04} introduce an adaptive distributed time
synchronization method for fully-connected Ultra Wideband (UWB)
networks. They use this as a basis for change detection
consensus. Wakamiya and Murata~\cite{wm04} propose a scheme for data
fusion in sensor networks where information collected by sensors is
periodically propagated without any centralized control from the edge
of a sensor network to a base station, using pulse-coupled
synchronicity. Wokoma et al.~\cite{wl02} propose a weakly coupled
adaptive gossip protocol for active networks. Each
of these applications clearly demonstrates the utility of
synchronicity as a primitive. However much of the prior work is
evaluated only in simulation and does not consider real communication
delay or loss.

Wireless radios exhibit non-negligible and unpredictable delays due to
channel coding, bit serialization, and (most importantly) backoff at
the MAC layer \cite{tpsn,ftsp}. In traditional CSMA MAC schemes, a
transmitter will delay a random interval before initiating
transmission once the channel is clear. Additional random (typically
exponential) backoffs are incurred during channel contention.  On the
receiving end, jitter caused by interrupt overhead and packet
deserialization leads to additional unpredictable delays.  Radio
contention deeply impacts the firefly model.  Multiple nodes
attempting to fire simultaneously will be unable to do so by the very
nature of the CSMA algorithm. As nodes achieve tighter synchronicity,
contention will become increasingly worse as many nodes attempt to
transmit simultaneously. The goal of this paper is to address the
limitations of current communication assumptions and realize a real
implementation of firefly-inspired synchronicity in sensor networks.

%% The simplicity of the pulse-coupled oscillator model and the parallel
%% between the limited capabilities of fireflies and sensor nodes,
%% provides an excellent motivation for investigating it as a
%% synchronicity algorithm for wireless sensor nodes. 

%% While their theoretical and simulation results are promising, their
%% model inherently assumes that radio communication is instantaneous,
%% and that nodes can instantaneously observe and react to neighbors
%% firing. They also assume that an arbitrary number of nodes can
%% transmit messages simultaneously when they fire.

%% Our work represents the first
%% implementation of firefly-inspired synchronicity on the MicaZ mote
%% hardware, and demonstrates the ability of the model to achieve tight
%% synchronicity given real radio and hardware limitations. 

