\section{Comparison to Other Methods}

Compared to algorithms such as RBS, TPSN and FTSP\cite{rbs,ftsp,tpsn},
the firefly-inspired algorithm represents a radically different
approach. All of the nodes behave in a simple and identical
manner. There are no special nodes, such as the root in TPSN or
reference node in RBS, that need to elected. A node does not maintain
any per-neighbor or per-link state; in fact it is completely agnostic
to the identity of its neighbors. The algorithm remains the same even
if the topology is multi-hop. There are no network-level
datastructures, such as the spanning tree in TPSN, that must be
re-established in case of topology change. As a result of these
properties, the algorithm is {\em implicitly robust to the
disappearance of nodes and links}. Lucarelli et al \cite{lucarelli04}
have shown that the algorithm works on time-varying topologies; our
testbed results show that the algorithm performs well even with
asymmetric, lossy links. The inherent adaptive nature of such
algorithms is one of the main attractions of biologically-inspired
approaches.

%% The goal of this paper was to determine whether such algorithms, based
%% on simple distributed control principles, were feasible in a the
%% wireless sensor network arena.

Nevertheless, it is not yet clear whether such an algorithm will be
competitive to algorithms such as TPSN and FTSP, in terms of accuracy
and overhead, and much work remains to be done. In terms of accuracy,
RFA achieves ~100 $\mu$sec which is significantly less than the
reported 10 $\mu$sec accuracy of FTSP, although as discussed before it
is difficult to make a clear comparison because of the errors caused
by using FTSP as our evaluation clock. We believe that the accuracy
can be increased to tens of microseconds by eliminating errors in our
evaluation methodology and by using a better optimized MAC-layer delay
estimation (as used in FTSP \cite{ftsp}).  However beyond that, the
accuracy will still be limited by clock skew, as discussed in Section
\ref{performance-evaluation}. We intend to investigate models that
synchronize both phase and frequency, which would eliminate errors
cause by clock skew. A second shortcoming of RFA is that the
communication overhead is high. In particular, we choose $T=1$ sec,
unlike FTSP which has a time period of 30 seconds. Assuming one could
compensate for clock skew, the main limit on $T$ is the time taken to
synchronize from startup. RFA takes approximately 200 time periods to
synchronize on MoteLab, which is significantly more than the diameter
of the network. On the other hand, the system recovers quickly from
small dispersive events. One option would be to use a simple
mechanism, such as an initial flood, to bring all nodes to within a
small phase difference quickly. Then RFA could operate at a much lower
frequency to tighten the accuracy and maintain synchronicity. A
different option is to allow nodes to asynchronously backoff, or
increase, their time period in multiples of $T$ depending on whether
they observe many out-of-phase firing events in their
neighborhood. Thus nodes would self-adjust the overhead.


%% Besides the above limitations, our algorithm also has some other
%% unique challenges, especially in terms of analysis. While the
%% algorithm is exceedingly simple, the analysis of its effect in a
%% distributed setting is not. Sophisticated mathematics is required to
%% prove convergence in a network \cite{lucarelli04} and it is difficult
%% to specify simple back-of-the envelop calculations on multi-hop
%% convergence time. Nevertheless, the inherent fault-tolerant and
%% adaptive quality of the algorithm is very attractive, and in that
%% respect it shares many characteristics with well-established
%% algorithms such as Ethernet/CSMA that are also based on simple node
%% adaptation strategies.



%% However, a much more difficult source of error is the
%% constant error introduced by clock skew. The nodes assume a default
%% fixed frequency, however because of clock skew their actual time
%% periods may differ by as much as .1 msec. Thus even if two nodes
%% completely agree on the phase, in the next time period they will fire
%% at different times. This causes the algorithm to constantly be
%% adjusting and in the end the effective frequency is in fact lower than
%% 1 sec. This lack of true agreement on frequency adds a constant error
%% to the process, and it is suggestive that our current accuracy of 100
%% microseconds is comparable to the expected skew on the clocks. One
%% approach to compensate would be to use models of distributed
%% synchronization that synchronize both phase and frequency - such as
%% models of how humans synchronize clapping [cite].

\section{Conclusions and Future Work}

In this paper we have presented a decentralized algorithm for
synchronicity, based on a mathematical model of synchronicity achieved
by biological systems. Our results show that even though the
theoretical models make simplifying assumptions, this technique still
works well and robustly in the face of realistic radio effects and
hardware limitations. In particular, we modified the algorithm to deal
with communication latencies, but it still achieved synchronicity
reliably and showed predictable behavior in relation to parameter
choice. In the future we intend to study in detail how lossy links,
clock skew and topology parameters affect the model. A detailed study
of the robustness of the system will also help us better understand
whether the adaptive and robust behavior of biological systems can be
leveraged to design robust algorithms for sensor networks.

\section{Acknowledgements}

Nagpal is supported by a Microsoft Faculty Fellowship, Patel is
supported by a NSF graduate Fellowship. We also thank Uri Braun, who
worked on initial stages of the project.
