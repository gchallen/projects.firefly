\section{Introduction}

Computer scientists have always looked to nature for inspiration.
Researchers studying distributed systems have long envied and attempted to
duplicate the fault-tolerance, lack of central control, and global
cooperation acheived in nature's systems.  Those of us studying sensor
networks also have every reason to be envious.  Designing software
coordinating the output of a collection of computation and communation
limited devices frequently feels as frustrating as orchestrating the activity 
of a colony of stubborn ants, or guiding a school of uncooperative fish.  And
yet ant colonies complete difficult tasks, fish flock travelling through the
sea, and along the banks of rivers in (somewhere I can't
remember) swarms of fireflies stretching for miles pulse in perfect unison.

Synchronicity is a powerful primitive to apply to the world of sensor
networks.  Evincing its utility requires distinguishing {\em synchronicity}
from {\em time synchronization}.  We define synchronicity as the ability to
organize {\em collective action} across a sensor network.  In contrast, time
synchronization or stamping is the ability to create a meaningful global time
base accesible to all sensor nodes in a network.  The two primitives are
independently useful.  Nodes within a sensor network may want compare the
time that they detected some event for beamforming or distributed signal
detection.  This task requires time stamping, but not collective action.
Other nodes may want to arrange communication schedules to permit efficient
duty cycling.  This task requires collective action, but not time stamping.
The two primitives are also complementary: nodes with access to a common time
base can schedule collective action, and, conversely, nodes who can arrange
collective action can establish a meaningful network-wide time base.  This
work is orthogonal to the large body of work in sensor networks concerning
time synchronization.  We address the utility and feasibility of
synchronicity.

Strogatz and Mirollo were the first to propose a mathematical model
explaining how fireflies and cardiac cells acheive stunning degrees of
synchronicity.  Recent work by Lucarelli and Wang relaxed one of the
assumptions in the original work, proving convergence across arbitrary
multi-hop topologies.  Our work is both a natural extension of and entirely
complementary to these earlier ideas.  In order to make theoretical arguments
the existing work cannot consider important vagaries of real hardware.  Both
Strogatz and Lucarelli assume instantaneous communication over perfect
bidirectional links, identical oscillators present on every node, and
arbitrary-precision floating-point arithmetic.  Each of these assumptions
falls down when faced with real sensor network devices.  Wireless devices
require medium access control which introduces significant communication
latencies.  Links between nodes are not perfect, not symmetric, and fluctuate
over time.  Crystal impurities cause noticeable skew between oscillators.
And the absense of floating-point units on typical sensor network device
processors renders floating-point calculations either computationally
expensive or impossible.

As the hardware exigencies above prove implementing Strogatz's algorithm
unchanged difficult if not impossible, we made a significant change to the
original theoretical model, which we present as the {\em reachback Firefly
algorithm}.  We will show that our simple change allows reasonable
communication delays to be accomodated and processing amortized.

We evaluate our algorithm on three fronts.  First we prove convergence of our
modified algorithm in simple cases, verifying that our change did not affect
the original algorithm's attractive properies.  Next we leverage TOSSIM, the
TinyOS simulator.  Using TOSSIM allows us to collect a large amount of data
exploring the parameter space across a wide range of network sizes and
topologies.  Finally we validate the simulator results by deploying our
implementation on a large sensor network testbed, demonstrating our algorithm
robust when faced with real radios and topological effects.  Our results show
that such a decentralized approach can provide reasonable levels of
synchronicity while incurring low overhead.

Our paper is organized as follows...
\XXXnote{MDW and RAD}
