\section{Evaluation Tools and Metrics}

Both our simulation and testbed experiments output a series of node
IDs and firing times. In order to discuss the accuracy of the achieved
synchronicity, it is necessary to identify groups of nodes firing
together.

For this purpose, we identify sets of node firings that fall within a
prespecified time window. We call each cluster of node firings a {\em
group}. Given a time window size $w$, the clustering algorithm outputs
a series of firing groups that meet two constraints. First, every node
firing event must fall within exactly one group.  Second, groups are
chosen to contain as many firing events as possible. 

We define the {\em group spread} as the maximum time difference
between any two firings in the group. The time window size $w$
represents the upper bound on the group spread. 

\subsection{Evaluation Metrics}

The two evaluation metrics that we are concerned with involve
the amount of time until the system achieves synchronicity (if at
all), and the accuracy of the achieved synchronicity.

\begin{description}
\item {\bf Time To Sync:} This is defined as the time that it takes
all nodes to enter into a single group and stay within that group for
9 out of the last 10 firing iterations. The value chosen for the time
window $w$ does impact the measured time to sync; a very small $w$
will result in a time to sync that is longer than with a larger $w$,
because it takes longer for all nodes to join a firing group within a
smaller time window. Also, as will be discussed in the next sections,
the simulator has lower time resolution than the testbed hardware
which means there is a limit on the accuracy it can
achieve. Therefore, for the simulator we set $w = 0.1$sec and in the
real testbed we set $w = 0.01$sec.

\item{\bf 50th and 90th Percentile Group Spread:} Recall that the
group spread measures the maximum time difference between any two
events in a firing group. We wish to characterize the {\em
distribution} of group spread for all groups after the system has
achieved synchronicity. Although synchronicity may be achieved
according to the time to sync metric above, we wish to avoid measuring
group spread while the system is still settling. Given the first sync
time $t_s$ and the time the experiment ends $t_e$, we calculate the
group spread distribution across all groups in the interval $[t_s +
\frac{(t_e - t_s)}{2}, t_e]$. In this way we are measuring the
distribution across all ``tight'' groups rather than settling
effects. We plot the 50th and 90th percentile of the distribution.

%To describe the level of
%synchronicity achieved by a given set of nodes, we proceed as follows.
%First, we examine only the time after which the system has synchronized, as
%defined by the time to sync.  To avoid penalizing systems which achieve
%tighter synchronicity than the binning parameter chosen we elide the first
%half of these spreads, thereby avoiding considering spreads produced when the
%system was still approaching its optimal level of synchronicity.  We
%calculate the 50th and 90th percentile group spread across the remaining
%data.  The meaning of these numbers is straightforward: over 50\% of the
%group firings the nodes were synchronized to at least the 50th percentile
%group spread, and likewise for the 95th percentile group spread.
\end{description}


Lastly, we define the {\em Firing Function Constant} (FFC) to be the
value $1/\epsilon$, which is the main parameter in the RFA
algorithm. As discussed in Section \ref{sec:params}, this parameter
limits the response of a node to be at most $T/FFC$ and
thus the time to synchronize is directly proportional to
$FFC$. 
