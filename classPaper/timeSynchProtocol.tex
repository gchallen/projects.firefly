
\section{A Firefly-Based Time Synchronization Protocol}

\subsection{Salient Features}
Our goal is to design a robust and scalable time synchronization scheme
based on the pulse-coupled oscillatory behavior observed in synchronized
fireflies for wireless sensor networks and evaluate its effectiveness
on a testbed of MicaZ sensor motes. 
In designing such an algorithm, {\bf Firefly Inspired Time Synchronization (FITS)}, 
our goals are:
\begin{enumerate}\addtolength{\itemsep}{-0.5\baselineskip}
\item {\bf Topology independence} : Should handle all kinds of partitioning
in sparse ad hoc networks.
\item {\bf Precision} : Ideally our algorithm will achieve the level of
precision attained by existing time synchronization protocols and thus the
error in time should be at most on the order of microseconds.
\item {\bf Scalability} : Should perform equally well in MoteLab, as it does 
with four sensors. 
\item {\bf Performance} : Low message overhead.
\end{enumerate}  


\subsection{From Synchronicity to Time Synchronization}
Synchronicity is the major challenge in time synchronization. Once all
motes fire together at regular intervals, the tasks of generating a
common timebase because significantly simplified. Moreover, the
majority of the error in time synchronization results from errors in
achieving this synchronicity -- or universal pulse.

Assuming the time synchronicity -- global heartbeat -- is good
enough, the task of providing time synchronization is nearly
trivial. All motes can use their global heartbeat to measure course
grained time. If necessary they could then use their local clocks for
intervals smaller than the standard pulse width. Additionally, it
should be relatively straightforward to translate between a reference
timebase and the collective FITS time. A could have knowledge of both
times by participating in FITS while also accessing the reference
timebase. This node should then be able to translate between the two
times.
