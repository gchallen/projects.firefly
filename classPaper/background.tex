
\section{Background and Related Work}
\label{sec:background}

There has been some work on designing algorithms to solve various sensor network
problems using pulse-coupled oscillator based schemes.
Hong and Scaglione~\cite{ssp03, tsrbc03} introduce an adaptive distributed time
synchronization method based on Strogatz and Mirollo's pulse-coupled 
oscillating system for Ultra Wideband (UWB) wireless ad hoc networks.  
Ultra Wideband (UWB) wireless devices can be used for short-range high-speed 
data transmissions suitable for broadband access to the Internet.
Recently Hong, Cheow and Scaglione~\cite{hcs04} propose a method to reach detection
consensus in massively distributed sensor networks that uses the synchronized
pulses of sensor nodes. In their proposal, synchronization updates are modeled 
by the dynamical evolution of a set of pulse-coupled oscillators which are 
guarateed to converge to synchrony in a variety of circumstances.  
An important drawback of their scheme is the reliance on an all-to-all communication
model which can cause instability on the synchronization scheme due to long
delays in signal propagation time and increased noise in the communication
channel.

Drawing from recent results in multiagent control, Lucarelli and Wang~\cite{lw04}
propose a simple method of synchronization without the all-to-all assumption 
mandated in the original firefly-based pulse coupled integrate and fire model
proposed by Mirollo and Strogatz.
Specifically, Lucarelli and Wang explore conditions on the update protocol that lead to synchrony
with bidirectional nearest neighbor coupling.  
One drawback of their approach is that they disregard factors such as noise 
in the detection model and propagation delay that will influence the accuracy of their algorithms
when tested in realistic wireless communication settings.  One of the contributions
of this paper is to evaluate the quality of the synchronicity achieved
in the presence of realistic constraints imposed in a wireless sensor setting
such as noise and collisions in the communication mediums, as well as clock skew
on individual sensors.

Wakamiya and Murata~\cite{wm04} propose a scheme for data fusion in sensor networks where information
collected by sensors is periodically propogated without any centralized control
from the edge of a sensor network to a base station as the propagation forms a concentric
circle. They use the pulse coupled oscillator model based on biological mutual
synchronization to allow sensor nodes to independently determine the cycle and timing
at which they emit information in synchrony. The nodes do this purely on the basis of 
observing radio signals emitted by sensor nodes in their vicinity. While they
consider robustness to node failure and energy efficiency, their evaluation is limited
to simulation and does not attempt to model factor such as noise and propagation delay
that exist in sensor networks, and will undoubtedly impact the effectiveness of 
their scheme in a realistic sensor network setting.

There has also been some work on using firefly synchrony theory to develop communication
protocols for wide area networks. In particular, Wokoma et al.~\cite{wl02} propose
a weakly coupled adaptive gossip protocol for application level active networks. 
Using the pulse coupled oscillator based scheme, they define a scheme for
management policy distribution and sychronization over a number of nodes in an application 
level active network. Their simulation results show that the algorithms are scalable, 
can work effectively in a realistic random network, and allow policy updates
to be distributed effectively. 

Sacks et al.~\cite{sb03} describe a firefly based synchronization scheme for 
sensor nodes in an environmental monitoring system. In their system, nodes
need a degree of synchronization as well as be able to adjust the sync-time base
to enable energy conserving operation. They accomplish this by allowing the nodes
to exhange messages in line with a flash interval, and \emph{lock} on to 
the message exchange phase in order to achieve synchronization.