\section{Conclusions and Future Work}
\label{sec:conclusion}

Designing a synchronicity scheme for wireless sensors
is particularly challenging in light of the resource and computation 
constraints on these devices. In this paper we
designed and implemented FICA, a firefly-based synchronicity mechanism 
for a wireless sensor platform.

A firefly based synchronicity scheme has several salient features.  
The underlying algorithm derived from Mirollo and Strogatz's original model 
of pulse-coupled integrate and fire dynamics is simple to implement and 
requires little parameter tweaking.  On the other hand, the evaluations
in this paper expose several weaknesses of this scheme. 
Our experiments indicate the sensitivity of our synchronicity scheme 
to factors such as the firing function constant and number of nodes in the network.
While the impact of node topology on the quality of synchronicity attained
is not surprising, it exposes yet another vulnerability of our scheme.
Furthermore, the scheme requires constant message exchange in the form 
of sending and receiving pulses, and makes strong demands on
bandwidth and energy, resources which are extremely constrained on sensor nodes.
Finally, it is not immediately clear how to make a number of pulse-coupled 
nodes synchronize to an external timescale such as GPS. This functionality,
if possible, could serve many different applications.

There are several avenues of future work that could be pursued. 
Our primary future goal is to resolve the issues we currently face with implementing
FICA on MicaZ nodes in Motelab, and be able to test the quality of our synchronicity
scheme in this real world setting.
It would be useful to perform a detailed study of the power requirements of 
implementing our synchronicity mechanism on wireless sensors and 
to quantitatively evaluate its operational demands.



