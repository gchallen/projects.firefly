\section{Introduction}
\label{sec:intro}

Advances in micro electro-mechanical systems technology in digital
circuit development, and in wireless communications are leading to smaller,
cheaper and lower power sensing and computing devices.  Radically new systems
are emerging that consist of thousands or even millions of tiny computing
devices interacting with the environment and communicating with each other.
The MicaZ mote and its descendants are current examples of research platforms
of this technology. Complex networks built from thousands of such devices are
expected to affect many aspects of our lives.  Some potential applications
include environmental monitoring, measuring continuously changing conditions
in a mobile commerce setting, smart office environment, and various tasks in
military tracking such as surveillance, target tracking, counter-sniper
systems or battlefield monitoring that propagates information to soldiers and
vehicles involved in combat.

Synchronicity amongst nodes in a wireless sensor network can play an important
role in many operations.  
Synchronicity can be used for a variety of general tasks such as providing time
synchronization, flexible power scheduling, and coordinated data logging.
Each of these tasks have many useful applications.  For instance time synchronization
in wireless sensor networks~\cite{rbs02,tpsn03,ftsp04}
can be used to provide information to determine locations of shooters in 
countersniper systems~\cite{sml04}, measure the time of flight of sound~\cite{ge01}; 
distribute a beamforming array~\cite{wym02}; form a low power TDMA schedule~\cite{ad98}; 
integrate a time-series of proximity detections into a velocity estimate~\cite{ce01}.
Flexible power scheduling can allow sensor nodes to perform their routine tasks while
reducing radio power consumption and supporting fluctuating power demand in the network~\cite{hdb04}.
Synchronous data logging can help keep data consistent by recognizing duplicate 
detection of the same event by different sensors~\cite{ige00}.
Providing synchronicity in wireless sensor networks is particularly useful
when fine time resolutions and a global clock federation, such as that provided
by the Flooding Time Synchronization Protocol (FTSP)~\cite{ftsp04} is not needed.
In such cases, for instance flexible power management schemes for sensor networks~\cite{hdb04},
and sensor network platforms for environment monitoring~\cite{sb03}, synchronicity
amongst network nodes is sufficient to fulfill the goals of the application.


Synchronicity has been observed in large biological swarms where
individuals apply quite simple transmission strategies. The canonical example
is the synchrony of fireflies observed in certain parts of southeast
Asia~\cite{ms90}.  The behavior of these systems can be modeled as a network
of pulse-coupled oscillators where each oscillator emits periodically a
self-generated pulse which will cause a coupling upon the pulsing time of
other oscillators. Peskin first introduced this idea in the context of cardiac
pacemakers~\cite{peskin75}. In the case of two oscillators, Peskin proved that the system
approached a state in which the oscillators fire synchronously under the assuption
that all oscillators have identical dynamics and each oscillator is coupled
to all the others, i.e. coupled \emph{all-to-all}.
Mirollo and Strogatz~\cite{ms90} provide one of the earliest complete
analystical studies of pulse coupled oscillator systems. They generalize 
Peskin's model and prove the synchronization of a network of all $N$ pulse-coupled
oscillators while retaining Peskin's assumptions, with the requirement
that the oscillators rise toward a threshold with a time-course which is 
monotonic and concave down.  Other models have also been considered by
making different assumptions on the propagagation delay and coupling
strength of the oscillators~\cite{a90,g96}.
The simplicity of Mirollo and Strogatz's model, and parallel between the limited
capabilities of fireflies and sensor nodes, provides excellent motivation
for investigating a pulse-coupled oscillatory system based synchronicity
algorithm in wireless sensor nodes.

Recent work by Lucarelli and Wang~\cite{lw04} demonstrates through proofs 
and simulations the effectiveness of this model while relaxing 
the all-to-all communication assumption.  Specifically, by casting the
pulse coupled equations in Peskin's model~\cite{peskin75} as a dynamical
system of the phase deviations, they derive a stability result based
on nearest neighbor coupling, and show convergence via simulation results 
with both static and time varying topologies.  Since all-to-all communication
is near impossible to attain in sensor networks, this work has tremendous
implications for the feasibility of implementing synchronicity using 
a pulse coupled oscillator based model in wireless sensor networks.
Lucarelli and Wang's work serves as our primary motivation for investigating
a pulse coupled integtrate and fire based synchronicity algorithm for 
sensor networks.

In this paper we design and implement {\bf FICA}, Firefly Inspired Collective Action,
a variant of Mirollo and Strogatz's synchronicity algorithm for the MicaZ 
wireless sensor platform. 
This method is based on a simple transmission strategy where nodes
integrate the coupling caused by the signal pulses received from other nodes,
and \emph{fire} a pulse, i.e. broadcast a message after reaching a designated threshold.
To our knowledge, this paper is the first attempt at evaluating the 
robustness of a pulse-coupled oscillator based synchronicity scheme in such 
real-world conditions.
Our implementation does not assume
all-to-all communication, and is tailored to the resource constraints of a wireless sensor
platform. 
We implement a linearized version of a logarithmic coupling procedure that satisfies 
the monotonicity and concavity requirements defined by Mirollo and Strogatz 
and thus theoretically guarantees to drive the phase differences to zero, and 
that can be computed using integer-only arithmetic.
We evaluate this algorithm on a collection of MicaZ sensors that are part of 
Motelab, as well as through extensive simulations on TOSSIM, the TinyOS mote simulator.


The specific contributions of this paper are:
\begin{itemize}\addtolength{\itemsep}{-0.5\baselineskip}
\item The design and implementation of FICA, a synchronicity algorithm for wireles sensor nodes that is based on Mirollo and Strogatz's pulse coupled oscillator model.
\item An evaluation of the accuracy and quality of the synchronicity attained by this algorithm when implemented in sensor nodes in both Motelab and TOSSIM.
\end{itemize}

The remainder of this paper is organized as follows. Section~\ref{sec:background} discusses
related work on time synchronization protols. Section~\ref{sec:fits} describes the
assumptions and intricacies of our firefly-based synchronicity algorithm, FICA,
and provides details of our implementation of this algorithm on the MicaZ 
wireless sensor platform. Section~\ref{sec:evaluation} describes the metrics we
use to evaluate the accuracy and quality achieved, shows results
of these metrics on TOSSIM and Motelab data, and discusses the circumstances
under which synchronicity amongst different sensors is achieved, as well as its
level of accuracy.  Section~\ref{sec:discussion} discusses some implementation
issues we faced in implementing FICA on Motelab.
Finally, Section~\ref{sec:conclusion} summarizes the key 
challenges involved and conclusions resulting from this work and makes
recommendations for future work.
 