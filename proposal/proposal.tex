\documentclass[8pt,twocolumn]{article}
\usepackage{ieee}
\usepackage{times}
\usepackage{array}
\usepackage{amsmath}
\usepackage{listings}
\usepackage{relsize}
\usepackage{color}
\newcommand{\ignore}[1]{}
\newcommand{\boldheading}[1]{{\vspace{0.1in}\noindent \bf #1} \hspace{0.06in}}
\newcommand{\otherheading}[1]{{\noindent \bf #1} \hspace{0.06in}}

\begin{document}
\title{Biologically Inspired Time Synchronization in Wireless Sensor Networks}
\author{ Uri Braun, Geetika Tewari, Geoffrey Werner-Allen\\ 
CS 263/266 Project Proposal\\
\{uribraun, gtewari, werner\}@eecs.harvard.edu }
\maketitle

\begin{abstract} 

Many wireless sensor network applications require time
synchronization for coordination, data logging, or event detection
purposes.
Providing this synchronization is challenging in the face of limited node
resources and communication bandwidth, node crystal frequency variations, and
lossy communication links across multi-hop networks.
Existing approaches are variants of approaches originally designed to work
well on far less constrained systems.

We propose to develop and evaluate Firefly
Inspired Time Synchronization (FITS), a protocol for distributed time
synchronization based on the pulse-coupled integrate-and-fire model
embedded in biological swarms.  Since FITS has evolved to work well on
multi-hop networks utilizing only local interactions, we think that it is a
promising approach to take to time synchronization in wireless sensor
networks. 
%The method is based on a simple transmission strategy where nodes integrate
%the coupling caused by the signal pulses received from other nodes, and fire
%a pulse after reaching a designated threshold. 
The requirements in designing such a protocol entail
that it should utilize low communication bandwidth, scale well for multi-hop
networks, and remain robust against topolgy changes and node failures.  We
intend to evaluate the performance of our protocol on the MicaZ
platform, using both TOSSIM -the TinyOS mote simulator, and Motelab, a
38~node sensor cluster deployed in Maxwell Dworkin.  We will evaluate its
performance in terms of power and bandwidth consumption and compare it with
Reference Broadcast Synchronization (RBS), an existing and commonly deployed
protocol for synchronization in wireless sensor networks.
%While there is significant previous work on time synchronization schemes in
%wireless sensor networks, and some work on pulse-coupled oscillating system
%based time synchronization schemes, 
To our knowledge this work is the only attempt at designing and
evaluating a pulse-coupled oscillator based time synchronization
protocol on the MicaZ platform.  \end{abstract}


\section{Introduction}

Advances in micro electro-mechanical systems (MEMS) technology in digital
circuit development, and in wireless communications are leading to smaller,
cheaper and lower power sensing and computing devices.  Radically new systems
are emerging that consist of thousands or even millions of tiny computing
devices interacting with the environment and communicating with each other.
The Mica2 mote and its descendants are current examples of research platforms
of this technology. Complex networks built from thousands of such devices are
expected to affect many aspects of our lives.  Some potential applications
include environmental monitoring, measuring continuously changing conditions
in a mobile commerce setting, smart office environment, and various tasks in
military tracking such as surveillance, target tracking, counter-sniper
systems or battlefield monitoring that propagates information to soldiers and
vehicles involved in combat.

Time synchronization in wireless sensor neworks is crucial in many
operations.  Time synchronization helps to keep data consistent by resolving
redundunt detection of the same event, it supports coordination and
communication, for example TDMA radio scheduling, and underpins common
services in distributed systems such as cryptography schemes, database
queries or distributed logging for debugging.  

Synchronization has been observed in large biological swarms where
individuals apply quite simple transmission strategies. The canonical example
is the synchronization of fireflies observed in certain parts of southeast
Asia~\cite{ms90}.  The behavior of these systems can be modelled as a network
of pulse-coupled oscillators where each oscillator emits periodically a
self-generated pulse which will cause a coupling upon the pulsing time of
other oscillators. Mirolla and Strogatz~\cite{ms90} were one of the first to
study this model and proved the synchronization of a network of pulse-coupled
oscillators under the assumption of uniform coupling and no propagation
delay.  The simplicity of this model, and parallel between the limited
capabilities of fireflies and wireless sensors, provides excellent motivation
for investigating a pulse-coupled oscillatory system based time
synchronization scheme.

\section{Background and Related Work}

Traditional time synchronization methods are ill suited in the framework of
wireless sensor networks and considerable attention has focussed on
developing more appropriate approaches.  Elson et al. identify the
requirements of an effective time synchronization protocol in the context of
wireless sensor networks as including energy efficiency, scalablility,
robustness and ad hoc deployment~\cite{wsn02}. They then propose design
principles for such schemes such as: use multiple tunable modes of
synchronization; do not maintain a global timescale for the entire network;
use post facto synchronization; adapt to application and exploit domain
knowledge.

The most widely adapted protocol used in the internet domain is the Network
Time Protocol (NTP) developed by Mills~\cite{mills94}.  NTP clients
synchronize their clocks to the NTP time servers with the accuracy in the
order of milliseconds. The time servers are synchronized by the external time
sources typically using GPS. While NTP has been proven to be effective in the
Internet, it is not well suited to sensor networks because of high message
overhead and its inability to adapt to topology change.

\subsection{Time Synchronization Protocols for Wireless Sensor Networks}

Three prominent examples of existing time synchronization protocols developed
for WSN are Reference Broadcast Synchronization (RBS)~\cite{rbs02},
Timing-sync Protocol for Sensor Networks (TPSN)~\cite{tpsn03} and Flooding
Time Synchronizing Protocol (FTSP)~\cite{ftsp04}. 

In RBS, a reference message is broadcasted at regular intervals.  The
receivers record their local time when receiving the reference broadcast and
exchange the recorded times with each other. While RBS has the advantage of
eliminating transmitter-side nondeterminism, it requires additional message
exchange to communicate the local time-stamps between nodes.  Karp et
al.~\cite{ogts03} extend RBS by introducing a model of the fundamental RBS
scheme that treats clock offset and clock skew on different scales. While
they provide a theoretical model for further refining RBS, it is not clear
whether it has any practical significance.
%Within the context of this model they derive the optimally precise and
%globally consistent time synchronization scheme.  Their approach can be used
%with any synchronization scheme that produces pairwise synchronization with
%independent errors.

TPSN~\cite{tpsn03} creates a hierarchical structure in the network and
coordinates pair-wise synchronization along the edges of the structure.  Each
node gets synchronized by exchanging two synchronization messages with its
reference node one level higher in the hierarchy. While the TPSN achieves two
times better performance than RBS by time-stamping the radio messages in the
Medium Access Control (MAC) layer of the radio stack, it fails to estimate
clock drift of nodes, which limits it's accuracy. Furthermore, it doesn't
support dynamic topology changes.

FTSP~\cite{ftsp04} utilizes periodic radio broadcast of synchronization
messages and implicit dynamic topology update. It achieves high precision
performance by using MAC-layer time-stamping, comprehensive error
compensation, including linear regression, which reduces time skew and keeps
network traffic overhead low.

In addition to the aforementioned, several other approaches exist.
Romer~\cite{tsan01} proposed a time synchronization scheme for sparse ad hoc
networks.  The basic idea of the algorithm is to generate time stamps using
unsynchronized local clocks, pass them between devices, and transform them to
the local time of the receiving device. The algorithm determines lower and
upper bounds for the real-time passed from generation of the time stamp in
the source node to arrival of the message in the destination node. It then
transforms these bounds to the time of the receiver and subtracts the
resulting values from the time of arrival in the destination node.  The
resulting interval specifies lower and upper bounds for the time stamp
relative to the local time of the receiving node.  A problematic aspect of
the algorithm is minimizing the amount of error in the interval.

On a theoretical note, Hu et al.~\cite{aots03} consider the time
synchronization problem under the assumption of asymptotic high node density
and vanishing per node power.  An interesting feature of their method is that
nodes collaborate to generate an aggregate waveform that can be observed
simultaneously by all nodes and that contains enough information to
synchronize all clocks.

\subsection{Pulse-Coupled Oscillator based Synchronization}

The idea of designing a time synchronization algorithm for WSNs inspired by
the synchrony observed in biological swarms has been explored in the past.
Hong et al.~\cite{ssp03, tsrbc03} introduce an adaptive distributed time
synchronization method based on a pulse-coupled oscillating system for Ultra
Wideband (UWB) wireless ad hoc networks.  Ultra Wideband (UWB) wireless
devices can be used for short-range high-speed data transmissions suitable
for broadband access to the Internet.
%UWB radio systems typically employ pulse modulation whereby extremely narrow
%pulses are modulated and emitted to convey or receive information. 
The emission bandwidths generally exceed one gigahertz.  Unlike our work,
this paper explores theoretical implications of this algorithm, and discusses
results of simulations rather than actual deployment on a testbed of wireless
sensors with UWB communication powers.

Recently Hong et al.~\cite{smrdc02} propose a method to reach detection
consensus in massively distributed sensor networks that uses the synchronized
pulses of sensor nodes.
%Initially each node makes a decision based on its local observation and
%broadcasts the binary decision to all other nodes. Then each node updates
%its decision based on the received information and broadcasts its decision
%again. This process is repeated until consensus is attained. 
The scheme is inspired by the mathematical models used to explain the
non-linear dynamics that lead to synchrony in certain natural swarms. The
convergence to consensus in their algorithm is guaranteed through the theory
of Mirollo and Strogatz~\cite{ms90}.

\section{Problem Statement}

\emph{Our goal is to design a robust and scalable time synchronization scheme 
based on the pulse-coupled oscillatory behavior observed in synchronized 
fireflies for wireless sensor networks and evaluate its effectiveness 
on a testbed of MicaZ sensor motes.} In designing 
our algorithm, {\bf Firefly Inspired Time Synchronization (FITS)}, our goals are:
\begin{enumerate}\addtolength{\itemsep}{-0.5\baselineskip}
\item {\bf Topology independence} : Should handle all kinds of partitioning
in sparse ad hoc networks.  
\item {\bf Precision} : Ideally our algorithm will achieve the level of
precision attained by existing time synchronization protocols and thus the
error in time should be at most on the order of microseconds.
\item {\bf Scalability} : Should perform equally well in MoteLab, as it does
with four sensors.  
\item {\bf Performance} : Low message overhead.
\end{enumerate}

\section{Research Methodology}

Our project will proceed in several distinct phases.  First, we must
understand what assumptions FITS makes that are unrealistic or unprovideable
in practice, and how and whether we can relax or approximate these
requirements.  Second, we will implement and test FITS using TOSSIM.  Third,
we will use TOSSIM to explore the parameter space discover the optimal
parameters.  Finally, we will deploy FITS on MoteLab and evaluate its
effectiveness next to RBS.

\subsection{Assumptions in the Firefly Model} The firefly synchronization
scheme makes two main assumptions that need to be given attention when
implementing the algorithm in sensor networks. First, it assumes that there
is no propagation delay; second, it assumes that all the fireflies can
hear each other simultaneously.  Obviously these assumptions do not hold in
any wireless sensor network.

We intend to address propagation delay within broadcast domains
by inserting a $\delta$ within each pulse message sent that will contain the
difference between the time that a message was intended to be sent and the
time when it was actually sent.  Each sensor node will then correct for this
time before it next fires.  In a multi-hop setting we need to design an
effective scheme to address this potential delay since in this case the delay
could grow to be very large.

There is no good way to address the requirement of full network connectivity,
and so this is one assumption that will have to be relaxed.  Given that in
situ biological synchronization such as that demonstrated by fireflies cannot
realistically rely on all-to-all communication, we are optimistic that this
assumption can be violated without harming the efficacy of the algorithm.

\subsection{Parameters to the Firefly Model} There are several parameters to
the firefly algorithm that need to be tweaked including: the $\epsilon$
amount moved on the concave function that represents when the firefly is
about to fire, and the choice of this concave function. The latter should
take into consideration that floating points are not available on the sensor
motes.

\subsection{Implementation Strategy} Initially, we intend to implement our
time sychronization on TOSSIM, the TinyOS simulator.  This is because TOSSIM
provides a C-like programming environment with convenient \emph{printf}-like
debugging statements that are not available on the sensor motes.  We will
need to modify TOSSIM to provide a realistic approximation of real
node-to-node clock drift, and we expect to parametrize this drift to allow
experimentation with synchronization algorithms under varying conditions.
The TOSSIM environment will allow us to conveniently change and fine-tune our
synchronization scheme.  On this platform we will evaluate the effectiveness
of our scheme according to the precision of time maintained, speed at
attaining synchronization, extent to which synchronization is attained,
stability of time synchronization, and robustness under changing topology.

Eventually we will evaluate the effectiveness of our technique in Motelab, a
38-node cluster of MicaZ motes deployed in Maxwell Dworkin.  Since this is a
fixed topology network, we will also obtain 5-10 motes, and experiment with
dynamic topologies.  

Evaluating power consumption and synchronization accuracy will require
external tools.  The former will be done by using an instrumented node on
MoteLab allowing current data to be collected at sampling rates of up to
3500Hz.  The latter is more complicated.  Evaluating the accuracy of the time
synchronization achieved will require two motes attached to GPS sensors
providing a 1-sec pulse with microsecond accuracy.  One mote will be used as
the synchronization seed and the other will allow the accuracy to be
evaluated, since they share a common time base.

\subsection{Comparison with RBS} We intend to compare our scheme with the
Reference Broadcast Synchronization (RBS) protocol. As shown in our timeline,
Table~\ref{table:timeline}, we intend to have working implementations of each
of our time synchronization schemes by the first week of December, and intend
to exchange data and conduct experiments on Motelab in the month of December.
In order to be able to compare performance of our algorithms effectively, we
have agreed on the following standards for experimentation:

\begin{enumerate}\addtolength{\itemsep}{-0.5\baselineskip}
\item {\bf Data logging and Results Format}
\item {\bf Mote communication frequencies} : rates at which the sensors
\emph{fire}, the rate of a GPS broadcast in RBS.
\item {\bf Types of Experiments} : details of conducting experiments on
Motelab.  
\item {\bf Evaluation parameters} :  message overhead, power
consumption, and multi-hop topology. 
\end{enumerate}

\begin{table*}[t]
\begin{center}
\smaller\smaller
\begin{tabular}{|l|l|}
\hline
{\bf October 26} & Project proposal due.  \\
\hline
{\bf November 4} & Learn to program the Telos motes with NesC. \\
	& Get acquainted with methods to exchange data, communicate radio messages, and log output on the motes. \\
\hline
{\bf November 9} & Design and implement a time synchronization scheme on Tossim.\\
        & Hack Tossim code to tweak Medium Access Control layer delays. \\
        & Conduct experiments implementing the time synchronization scheme, and fine-tune it. \\
\hline
{\bf November 23} & Extend the design to Motelab, the 38 sensor node network in Maxwell Dworkin.\\
        & Decide changes to synchronization scheme necessary to make it multi-hop. \\
        & Conduct experiments on Motelab, and fine-tune the multi-hop algorithm.\\

\hline
{\bf December 6} & Collect data and exchange data with Geoff Werner-Allen. \\
	& Compare the results of our time synchronization with Geoff's implementation of RBS.\\
        & Begin first draft of paper focusing on research methodology and implementation techqniques.\\
\hline
{\bf December 16} & Project presentations. \\
	&   Report results of experimental analysis of our time sychronization scheme, and its performance in comparison with RBS. \\
	&   Explore variants of our time synchronization scheme. \\
	&   Determine new performance metrics for comparison. \\
	&   Perform more experiments to compare performance. \\
\hline
{\bf December 30th} & Continue writing the paper focusing on results and evaluation section. \\
	&  Continue experiments to gather more results. \\
\hline
{\bf January 5th} & Begin preparing for presentation. \\
	&   Begin work on final draft of paper.\\
\hline
{\bf January 10th} & Submit final paper.\\
\hline
\end{tabular} 
\caption{Our Research Timeline}
\label{table:timeline}
\end{center} 
\end{table*} 


\section{Research Schedule}
We have compiled a list of intermediate deadlines for different 
components of our project shown in Table~\ref{table:timeline}. We consider 
our timeline to be reasonable and realistic.  After a detailed understanding of the TinyOS and TOSSIM
codebase and programming methodology, which we intend to acquire over the next two weeks, 
we plan to allocate individual implementation tasks between us.

\section{Future Work}
Given the limited prior art in applying biologically motivated time synchronization mechanisms
in the context of sensor networks, we intend to make this work the basis for a 
conference publication.

\smaller
\bibliographystyle{abbrv}
\bibliography{proposal}

\end{document}
